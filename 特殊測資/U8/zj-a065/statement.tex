% 題目來源:https://zerojudge.tw/ShowProblem?problemid=a065 
% 黃惟:https://yuihuang.com/zj-a065 
% 品python:https://hackmd.io/@10946009/zj-a065 
文文記性不太好,常常會忘東忘西。\\
他也常忘記提款卡密碼,每次忘記密碼都得帶著身份證、存摺、印章親自到銀行去重設密碼,還得繳交 50 元的手續費,很是麻煩。\\
後來他決定把密碼寫在提款卡上免得忘記,但是這樣一來,萬一提款卡掉了,存款就會被盜領。\\
因此他決定以一個只有他看得懂的方式把密碼寫下來。\\
\\
他的密碼有 6 位數,所以他寫下了 7 個大寫字母,相鄰的每兩個字母間的「距離」就依序代表密碼中的一位數。\\
所謂「距離」指的是從較「小」的字母要數幾個字母才能數到較「大」字母。\\
字母的大小則是依其順序而定,越後面的字母越「大」。\\
\\
假設文文所寫的 7 個字母是 POKEMON,那麼密碼的第一位數就是字母 P 和 O 的「距離」,由於 P 就是 O 的下一個字母,因此,從 O 開始只要往下數一個字母就是 P 了,所以密碼的第一位數就是 1。\\
密碼的第二位數則是字母 O 和 K 的「距離」,從 K 開始,往下數 4 個字母 (L, M, N, O) 就到了 O,所以第二位數是 4,以此類推。\\
因此,POKEMON 所代表的密碼便是 146821。\\
\\
噓!你千萬別把這個密秘告訴別人哦,要不然文文的存款就不保了。\\
