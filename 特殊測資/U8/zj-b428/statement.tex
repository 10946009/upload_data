% 題目來源:https://zerojudge.tw/ShowProblem?problemid=b428 
% 黃惟:https://yuihuang.com/zj-b428 
% 品python:https://hackmd.io/@10946009/zj-b428 
背景曾幾何時,基礎題庫已經成了不基礎的題庫。\\
小小新手們寫個題目,不少拿了 TLE、CE 求助無門,就再也不想打開 Zerojudge。\\
高中生哪有寫這麼困難的題目,高中生都不像高中生。\\
在某 M 那個年代寫的題目非常簡單,沒有特別變化處理,更別說多麼高檔的資料結構,暴力算法 (naive algorithm) 就能輕鬆切題。\\
「年代變了呢,現在的高中生要寫出比大學生的某 M 更困難的題目」重溫解題的那份初心吧!題目描述 在西元前就存在的一種加密-凱薩加密為目前最早發現的替換加密 (substitution cipher)。\\
其原理很簡單,將一段明文往替換成往後數的第 $k$ 個英文字母。\\
若用數學式表示凱薩加密和解密,如下:加密 $C = E_K(P) = (P + k) \mod 26$解密 $P = D_K(P) = (C - k) \mod 26$\\
 例如 $k = 3$ 時,發生的情況如下:明文字母表:ABCDEFGHIJKLMNOPQRSTUVWXYZ 密文字母表:DEFGHIJKLMNOPQRSTUVWXYZABC從數學的觀點來看,每一個字母就是一個數字。\\
A = 0, B = 1, C = 2, ...,X = 23, Y = 24, Z = 25