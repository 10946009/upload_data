% 題目來源:https://zerojudge.tw/ShowProblem?problemid=c085 
% 黃惟:https://yuihuang.com/zj-c085 
% 品python:https://hackmd.io/@10946009/zj-c085 
電腦無法產生真正的亂數(Random numbers),但是經由某些程序電腦可以產生虛擬亂數(pseudo-random numbers)。\\
亂數被使用在很多應用上,像是模擬等。\\
\\
有一種常用的虛擬亂數產生方法:如果上一個亂數是L,那下一個亂數產生的方法是 (Z*L+I) mod M,在這裡Z、I、M都是常數。\\
例如:假設Z=7 I=5 M=12。\\
如果第一個亂數(通常叫做 seed)是 4 , 那我們可以產生以下幾個虛擬亂數:\\
\\
我們可以發現,經過6個數字後,虛擬亂數的序列重複了,也就是說cycle length=6。\\
 在這個問題中,你將會被給Z、I、M還有L(就是seed)的值(全部不大於9999),對每一組Z、I、M、L,要請你輸出所產生的虛擬亂數的cycle length。\\
 請注意:cycle並不一定從seed開始。\\
