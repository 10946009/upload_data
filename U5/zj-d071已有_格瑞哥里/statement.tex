% 題目來源:https://zerojudge.tw/ShowProblem?problemid=d071 
% 黃惟:https://yuihuang.com/zj-d071 
% 品python:https://hackmd.io/@10946009/zj-d071 
現行的曆法是從羅馬人的曆法演變而來的。\\
凱撒大帝編纂了一套曆法,後人稱之為儒略曆 (Julian calendar)。\\
在這曆法中,除了四、六、九、及十一月有30天,二月在平年有28天,在閏年有29天以外,其他的月份都是31天。\\
再者,在這曆法中,每四年有一個閏年。\\
這導因於古代羅馬的星象學家算出一年有365.25天,因此每隔四年就要加一天以保持曆法和季節的一致。\\
於是,他們就在四的倍數的年份多加了一天 (二月29日)。\\
\\
儒略法:\\
四的倍數的年份均為閏年,這年會多一天 (二月29日)。\\
\\
在1582年,教宗格瑞哥里 (Gregory) 的星象學家發現一年並不是365.25天,而是比較接近365.2425天。\\
因此,閏年的規則便修正如下:\\
格瑞哥里法:\\
除了不是400的倍數的100的倍數以外,四的倍數的年份均為閏年。\\
\\
為了要彌補截至當時季節和日曆已產生的誤差,當時的日曆便往前挪移了10天:1582年10月4日的第二天為10月15日。\\
\\
英格蘭和它的帝國 (包括美國) 一直到1752年才改用格瑞哥里曆,當年的9月2日的第二天為9月14日。\\
(未同步採用新曆乃肇因於亨利八世和教宗的惡劣關係。\\
)\\
請依現行的曆法判斷所給的西元年份是平年還是閏年。\\
