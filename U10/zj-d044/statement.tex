% 題目來源:https://zerojudge.tw/ShowProblem?problemid=d044 
% 黃惟:https://yuihuang.com/zj-d044 
% 品python:https://hackmd.io/@10946009/zj-d044 
在1949年印度數學家 D.R Kaprekar發現了一種數字:Self-numbers。\\
對任何正整數 n ,定義d(n)為n加上其各數字的和。\\
例如:d(75)=75+7+5=87。\\
給任一個正整數 n 當作一個起始點,你可以產生無限的數字序列:n, d(n), d(d(n)), d(d(d(n))),…例如:如果你從33開始,下一個數字是33+3+3=39,再下一個數字是39+3+9=51,再下一個數字是51+5+1=57。\\
所以你可以產生以下的序列:33, 39, 51, 57, 69, 84, 96,111, 114, 120, 123, 129, 141, ……我們稱n為d(n)的generator。\\
在上面的例子中33是39的generator,39是51的generator,51是57的generator,以下類推。\\
有些數有不只一個generator,例如:101有2個generators,91和100。\\
如果一個數沒有generator,那他就是一個self-number。\\
比100小的self-number:1, 3, 5, 7, 9, 20, 31, 42, 53, 64, 75, 86, 97本問題是:找出所有小於或等於1000000的self-numbers。\\
